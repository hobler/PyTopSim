The configuration file \texttt{xxx.cfg} is read using Python's 
\texttt{ConfigParser} module. The syntax therefore has to adhere to the 
\texttt{ConfigParser} rules, which basically demand the configuration file
to be written in the Windows INI format, like
%
\begin{verbatim}
   # A comment
   [section1]
   option1 = value1
   option2 = value2

   [section2]
   option1 = value3
   option3 = value4
   ...
   \end{verbatim}
%
Note that same option names appearing in different sections will be
distinguished.

One pitfall that requires attention when writing a configuration file is that
TopSim will check the type of the option values against its defaults. Since
Python distinguishes between tuples (lists) and scalar values, it is essential
that tuples are specified as tuples and scalars (e.g. floats) as scalars
(floats). A tuple parameter can be specified as
%
\begin{verbatim}
   option = (item1,item2,...,itemn)
\end{verbatim}
%
with the dots replaced by other item values. The parenthesis may be
omitted, but the commas are essential. Several TopSim parameters are tuples,
among them parameters that have components in x- and y-direction. While in a 2D
simulation only the x-component is used, the parameter still is a tuple and must
be specified as a tuple like
%
\begin{verbatim}
   option = item1,
\end{verbatim}
%
Incomplete tuples will have their missing elements duplicated from the last
given component. If in doubt, consult the \texttt{[Output]LOG\_FILE}, where a
copy of all parameters after any substitutions can be found.
