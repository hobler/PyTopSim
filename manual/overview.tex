TopSim is a 2D/3D \textbf{Top}ography \textbf{Sim}ulator. The main
application in mind are ion beam induced processes such as sputtering and ion beam assisted
etching and deposition. While the 3D mode has been tested previously, it
has not been used recently and might be broken. So, at the moment, use 3D at
your own risk.

TopSim is written in Python with the numpy extension using several
function from scipy. Therefore, execution of TopSim requires
python, numpy, and scipy to be installed. In addition, TopSim includes simple
facilities for displaying intermediate and final surfaces, based on matplotlib.
If this feature is used, matplotlib has to be installed as well.

The simplest way to run TopSim is to type
%
\begin{verbatim}
	python TopSim.py xxx.cfg
\end{verbatim}
%
in the source directory of TopSim. \texttt{xxx.cfg} is the name of a
configuration file (replace \texttt{xxx} with an actual file name), which holds
the input parameter specifications. A more convenient way is to run it from a
separate directory. In that case the path to TopSim has to be specified: 
%
\begin{verbatim}
	python path_to_topsim/TopSim.py xxx.cfg
\end{verbatim}
%
To view the results right after the end of the simulation, set the
\texttt{DISPLAY\_SURFACE} of the \texttt{Output} section to \texttt{True}. A
window will open at the end of the simulation, and a short help text will be
written to the terminal, explaining the meaning of keystrokes you can enter
when focused on the plot window. The same functionality is provided by the
\texttt{plot.py} script, which may be called separately from TopSim like
%
\begin{verbatim}
	python path_to_topsim/plot.py xxx.srf
\end{verbatim}
%
\texttt{xxx.srf} denotes the name of the surface file which has been written by
the TopSim run with the \texttt{xxx.cfg} configuration file. Both the plot
function of TopSim and \texttt{plot.py} will look for a file
\texttt{baseline/xxx.srf} and, if present, plot it with dashed lines for
comparison.

This manual is a stub. More documentation is hoped to come. However, we try to
keep the description of the input parameters complete and accurate.
