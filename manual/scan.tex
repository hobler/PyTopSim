\begin{keydescription}{\texttt{DWELL\_TIME}}
This parameter specifies the dwell time per pixel in seconds. If
\texttt{TYPE='pixel file'} the dwell time is multiplied by the value found in
the \texttt{PIXEL\_FILE}. The total simulation time in case of a raster or
serpentine scan is given by $\mathtt{PASSES} \times \mathtt{PIXELS[0]} \times
\mathtt{PIXELS[1]} \times \mathtt{DWELL\_TIME}$. If \texttt{DWELL\_TIME} is not
specified, it is calculated from \texttt{[Beam]TOTAL\_TIME}. If both
\texttt{DWELL\_TIME} and \texttt{[Beam]TOTAL\_TIME} are specified,
\texttt{[Beam]TOTAL\_TIME} will be overwritten by the value calculated from
\texttt{DWELL\_TIME}.
\begin{keytab}
   Type:    \> real \\
   Default: \> required if \texttt{[Beam]TOTAL\_TIME} is not specified
\end{keytab}
\end{keydescription}

\begin{keydescription}{\texttt{OVERLAP}}
This parameter specfies whether all the beams of a pass should be overlapped to
a single beam. The advantage of an overlapped beam is that the time step is not
limited to the pixel dwell time. This approximation is only valid if the change
in the surface caused by a single pixel during \texttt{DWELL\_TIME} is small
enough.
\begin{keytab}
   Type:    \> boolean \\
   Default: \> \texttt{False} 
\end{keytab}
\end{keydescription}

\begin{keydescription}{\texttt{OVERLAP\_Y}}
This parameter specfies whether all the beams along the line of a scan in
y-direction should be overlapped to a single beam. The advantage of an
overlapped beam is that the time step is not limited by the pixel dwell time.
This approximation is only valid if the change  in the surface caused by a
single pixel is small enough. 
\begin{keytab}
   Type:    \> boolean \\
   Default: \> \texttt{False} 
\end{keytab}
\end{keydescription}

\begin{keydescription}{\texttt{Y\_OVERSCAN}}
This parameter instructs the simulation to scan each Y line in a 3D 
simulation a preset number of times. This is used to produce a marching
line scane in 3D mode without using overlap modes. A value of 1 indicates
that the line is to be scanned once and then the beam will step in the x 
direction. A larger value causes the beam to scan that Y line a larger
number of times before stepping in the x direction. 
\begin{keytab}
   Type:    \> integer \\
   Default: \> 1 
\end{keytab}
\end{keydescription}

\begin{keydescription}{\texttt{PASSES}}
This parameter specifies the number of passes, i.e. the number of times the
pattern defined by \texttt{TYPE} is repeated.
\begin{keytab}
   Type:    \> integer \\
   Default: \> 1
\end{keytab}
\end{keydescription}

\begin{keydescription}{\texttt{PIXEL\_FILE}}
This parameter specifies the name of the pixel file, containing a list of
pixels coordinates and dwell times. The dwell times will be multiplied with
\texttt{DWELL\_TIME}.
\begin{keytab}
   Type:    \> character string \\
   Default: \> required if \texttt{TYPE='pixel file'}
\end{keytab}
\end{keydescription}

\begin{keydescription}{\texttt{PIXEL\_SPACING}}
This parameter specifies the pixel spacing in x- and y-direction for
\texttt{TYPE='raster'} and \texttt{TYPE='serpentine'}. If only one value of
\texttt{PIXEL\_SPACING} is given, the second value is assumed to be identical to
the first value. Ignored if \texttt{TYPE='none'} or \texttt{TYPE='pixel file'}.
The units of \texttt{PIXEL\_SPACING} are nm.
\begin{keytab}
   Type:    \> (real,) \\
   Default: \> \texttt{0.,0.} if \texttt{TYPE='none'} \\
            \> \texttt{SCAN\_WIDTH/max(PIXEL-1,1)} otherwise
\end{keytab}
\end{keydescription}

\begin{keydescription}{\texttt{PIXELS}}
This parameter specifies the number of pixels in x- and y-direction for
\texttt{TYPE='raster'} and \texttt{TYPE='serpentine'}. If only one value
of \texttt{PIXELS} is given, the second value is assumed to be 1. Ignored if
\texttt{TYPE='pixel file'}.
\begin{keytab}
   Type:    \> (integer,) \\
   Default: \> \texttt{1,1} 
\end{keytab}
\end{keydescription}

\begin{keydescription}{\texttt{SCAN\_WIDTH}}
This parameter specifies the scan width in x and y direction for
\texttt{TYPE='raster'} and \texttt{TYPE='serpentine'}. If only one value of
\texttt{SCAN\_WIDTH} is given, the second value is assumed to be identical to
the first value. Ignored if \texttt{TYPE='pixel file'}. The units of
\texttt{SCAN\_WIDTH} are nm.
\begin{keytab}
   Type:    \> (real,) \\
   Default: \> \texttt{PIXEL\_SPACING*max(PIXELS-1,1)} if
               \texttt{PIXEL\_SPACING} is specified \\ 
            \> \texttt{1.,1.} otherwise
\end{keytab}
\end{keydescription}

\begin{keydescription}{\texttt{TYPE}}
This parameter specifies the scan type. \texttt{TYPE='raster'} means a scan
along parallel lines starting each line on the the same side.
\texttt{TYPE='serpentine'} means a scan along parallel lines starting every
other line on the the same side. \texttt{TYPE='pixel file'} means the pixel positions
are specified in the \texttt{PIXEL\_FILE}. For \texttt{TYPE='none'} no scan is
performed, being equivalent to \texttt{TYPE='raster'} and \texttt{PIXELS=1,1}.
\begin{keytab}
   Type:    \> character string \\
   Default: \> \texttt{'raster'} \\
   Range:   \> \texttt{'none'}, \texttt{'raster'}, \texttt{'serpentine'},
               \texttt{'pixel file'}
\end{keytab}
\end{keydescription}
