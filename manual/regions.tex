Regions define the geometry of spatial domains for the definition of material
properties (see the \texttt{Physics]} section).

\begin{keydescription}{\texttt{NUMBER\_OF\_REGIONS}}
This parameter specifies the number of regions. (soon to be depreciated)
\begin{keytab}
   Type:    \> integer \\
   Default: \> required
\end{keytab}
\end{keydescription}

\begin{keydescription}{\texttt{REGIONS}}
This parameter specifies the boundaries of the regions. (not yet used). [This has been replaced 
in 2D by the concept of primitives and domains]
\begin{keytab}
   Type:    \> real \\
   Default: \> \texttt{($-\infty$,$\infty$)}
\end{keytab}
\end{keydescription}

\begin{keydescription}{\texttt{DOMAINS}}
This parameter specifies a dictionary that is used to define combinations of primitives. 
Currently the only Domains that can be used are material domains. The indexed order of the 
material names is specified with the \texttt{Physics.MATERIAL\_NAMES} parameter. Each entry to 
the dictionary is made up of two strings: the key, and the equation. The key is the name of the 
Domain, this name should match one of the materials used in  \texttt{Physics.MATERIAL\_NAMES}, 
unmatched keys will be ignored. The equation instructs the simulator on how to combine spatial primitives. Equations have two operations +(plus) and -(minus), plus indicates the union of two 
spatial primitives, minus indicates the subtraction of the second primitive from the first. All 
domain equations are evaluated from left to right such that a+b-c = (a+b)-c, the leading 
operator is assumed to be positive and any attempt to use unary operators will result in an 
error. These operations (+/-) operate on spatial primitives as defined by their keys, 
the spatial primitives are specified with the \texttt{PRIMITIVES} parameter.  
\begin{keytab}
   Type:    \> dictionary \\
   Default: \> \texttt{($-\infty$,$\infty$)}
\end{keytab}
\end{keydescription}

\begin{keydescription}{\texttt{PRIMITIVES}}
This parameter specifies the spatial primitives used in the simulation. Format: key: primitive-parameters.
\begin{keytab}
   Type:    \> real \\
   Default: \> \texttt{($-\infty$,$\infty$)}
\end{keytab}
\end{keydescription}
