\begin{keydescription}{\texttt{ANGLE\_BISECTOR}}
This parameter specifies whether the angle bisector is used to determine the 
direction of point advancement. If \texttt{ANGLE\_BISECTOR=False}, point
advancement is perpendicular to the line connecting the two neighboring points.
Only valid if \texttt{SURFACE\_TYPE='rectilinear'}.
\begin{keytab}
   Type:    \> boolean \\
   Default: \> \texttt{False}
\end{keytab}
\end{keydescription}

\begin{keydescription}{\texttt{DEPOSITION}}
This parameter enables beam induced deposition.
\begin{keytab}
   Type:    \> boolean \\
   Default: \> \texttt{False}
\end{keytab}
\end{keydescription}

\begin{keydescription}{\texttt{DIMENSIONS}}
This parameter specifies the number of spatial dimension of the coordinate
system.
\begin{keytab}
   Type:    \> integer \\
   Default: \> required \\
   Range:   \> \texttt{1}, \texttt{2}, \texttt{3}
\end{keytab}
\end{keydescription}

\begin{keydescription}{\texttt{ETCHING}}
This parameter enables beam induced etching.
\begin{keytab}
   Type:    \> boolean \\
   Default: \> \texttt{False}
\end{keytab}
\end{keydescription}

\begin{keydescription}{\texttt{INTERPOLATE}}
This parameter specifies whether the grid points are interpolated back to the 
x- and y-positions of the initial grid. Only valid if
\texttt{SURFACE\_TYPE='rectilinear'}.
\begin{keytab}
   Type:    \> boolean \\
   Default: \> \texttt{True}
\end{keytab}
\end{keydescription}

\begin{keydescription}{\texttt{ISOTROPIC}}
This parameter specifies whether isotropic sputtering is used or the sputter 
yield is read from a table (see the \texttt{[Physics]} section). If
\texttt{ISOTROPIC=True}, the first order sputter fluxes are set equal to the
beam flux. Otherwise, the first order sputter fluxes are calculated as the
product of the beam fluxes, the cosine of the incidence angle, and the sputter
yield.
\begin{keytab}
   Type:    \> boolean \\
   Default: \> \texttt{False}
\end{keytab}
\end{keydescription}

\begin{keydescription}{\texttt{REDEP\_1}}
This parameter specifies whether first order redeposition is considered, i.e.
redeposition of atoms ejected by the beam (first order sputtering).
\begin{keytab}
   Type:    \> boolean \\
   Default: \> \texttt{False}
\end{keytab}
\end{keydescription}

\begin{keydescription}{\texttt{REDEP\_2}}
This parameter specifies whether second order redeposition is considered, i.e.
redeposition of atoms ejected in second order sputtering.
\begin{keytab}
   Type:    \> boolean \\
   Default: \> \texttt{False}
\end{keytab}
\end{keydescription}

\begin{keydescription}{\texttt{SPUTTER\_2}}
This parameter specifies whether second order sputtering is considered, i.e.
sputtering caused by backscattered ions.
\begin{keytab}
   Type:    \> boolean \\
   Default: \> \texttt{False}
\end{keytab}
\end{keydescription}

\begin{keydescription}{\texttt{SHADOWING}}
This parameter specifies whether complex shadowing, in addition to simple 
visibility, is applied to the simulation of the surface. It is currently only
availabe in 2D.
\newline
Simple visibility finds points that are shadowed by the limits of ejection, i.e.
the ejection angles cannot be larger than 90 degrees.
\newline
Complex shadowing finds points that are shadowed by other points being in the way
by finding local maxima in the cosine (Source to Destination). Such a Point of
Interest is the starting point of a shadow, and the next point with a higher cosine
is the corresponding endpoint.
\begin{keytab}
   Type:    \> boolean \\
   Default: \> \texttt{False}
\end{keytab}
\end{keydescription}

\begin{keydescription}{\texttt{SURFACE\_TYPE}}
This parameter specifies the type of the grid for the surface. 
\begin{keytab}
   Type:    \> character string \\
   Default: \> \texttt{'rectilinear'}
\end{keytab}
\end{keydescription}
